\section{Desarrollo}

\subsection{kNN}

El algorimo kNN (k Nearest Neighbours) se basa en el análisis de un conjunto de puntos del espacio para determinar a qué clase corresponde el nuevo objeto. En este caso cada imagen estara representada como un vector donde cada elemento es un píxel distinto de la misma. El conjunto de puntos elegido serán aquellas $k$ imágenes que más cerca se encuentren de la imágen a clasificar. Lo que se hace es clasificar a la nueva imágen como la clase que mayor representantes tenga en este conjunto de puntos cercanos.

Este algoritmo puede ser sumamente costoso en cuanto al tiempo de cómputo, y si la dimensión de los puntos a clasificar es muy grande, hacer uso del mismo podría resultar impracticable. Es por esto que se implemento un método cuyo objetivo es preprocesar las imágenes para reducir la cantidad de dimensiones de las muestras, permitiendo a kNN trabajar con muestras con una cantidad de variables menor. Este método es conocido como PCA (Principal components analysis).

\subsection{PCA}

El método de análisis de componentes principales se encarga de cambiar de base el conjunto de datos de entrada, reduciendo la dimensión de cada elemento tanto como se desee. Al reducir la dimensión de un punto es claro que se pierde determinada información sobre el mismo, pero la particularidad de este método es que, como su nombre lo indica, se queda con las componentes más importantes, dejando de lado las que menos información aporten. De este modo, la información que se descarta es la que menor relevancia tiene, por lo que se lo considera una buena manera de reducir el espacio de la entrada.

Es importante aclarar que el método no solamente reduce la dimensión de los datos, sino que cambia la base de los mismo. Por lo que es necesario tener esto en cuenta al momento de trabajar con los mismos. Si se utiliza este método para reducir la entrada de kNN, es necesario cambiar a la misma base la imagen a clasificar, de lo contrario estarían comparándose elementos que viven en distintos espacios.

\bigskip

Procedimiento para el cambio de base:

\begin{enumerate}
\item Siendo $X$ la matriz cuyas filas contienen las imágenes, se calcula $\mu = (x_1 + ... + x_n)/n$ el promedio de las imágenes, donde $x_i$ es la $i$-ésima imagen.
\item Se crea la matriz $X_\mu$ que contiene en la $i$-ésima fila al vector $(x_i - \mu)^t$
\item Se calcula la matriz de covarianza $M = X_\mu^tX_\mu$
\item Se calculan los autovectores de $M$ mediante el método de las potencias, con deflación. Cómo cada iteración en la que calculamos el autovector de la matriz en cuestión, este está asociado al autovalor de máximo módulo, los autovectores que habremos calculado se encontrarán ordenados por relevancia. De esta manera se calculan tan solo $\alpha$ autovectores, con $1 \geq \alpha \geq n$, siendo $\alpha$ la dimensión a la que se quiere reducir las imágenes.
\item Se contruye la matriz $V$ con los autovectores caluclados previamente, dispuestos como columnas. $V$ es la matriz de cambio de base.
\item Por último, se reduce la dimensión de $X$ cambiando su base. El resultado final es $V^tX^t$ que contiene la misma cantidad de imágenes pero expresadas en otra base, y en lugar de tener dimensión $n$, cada una tiene dimensión $\alpha$.
\end{enumerate}\tabularnewline

Puede que sea interesante notar que una vez hecho el cambio de base de las imágenes, éstas representan a las imágenes pero si se trata de graficarlas se obtendrá algo muy distinto a lo que era anteriormente y no podrá visualizarse nada en concreto. Solo volviendo a la base original sería posible, pero ya se habrá perdido mucha información por lo que probablemente sea difícil encontrarlo útil.

\subsection{Data augmentation}

Los métodos implementados y algoritmos similares, su performance está muy relacionada a la calidad y cantidad de datos. Por este motivo se nos ocurrió que aumentar la cantidad de datos podría ser una buena idea. Tenemos la hipótesis que al realizar un “Data Augmentation”, agregando datos que podrían llegar a ser parámetros de entrada para clasificar(en nuestro caso dígitos manuscritos) obtendremos mejores resultados.

Para agrandar la cantidad de datos, decidimos aplicar transformaciones a las imágenes que ya tenemos etiquetadas, y si esta transformación no es lo suficientemente abrupta, la imagen transformada debería seguir siendo el mismo dígito.

Al ser dígitos manuscritos, los trazos no son perfectos, por lo que tratamos de encontrar alguna forma de aplicar transformaciones que nos generan ciertas irregularidades para imitar este comportamiento. Inspirandonos en este paper (http://citeseerx.ist.psu.edu/viewdoc/download?doi=10.1.1.160.8494\&rep=rep1\&type=pdf)
usamos una técnica que se llama deformaciones elásticas, y las implementamos en matlab generando un desplazamiento aleatorio de cada pixel y después aplicando un filtro gausseano para suavizar este desplazamiento. 

Con este método generamos un nuevo data-set aplicando esta función a cada imagen y así ahora tenemos 84000 datos. Lamentablemente perdimos los resultados de este experimento pero pudimos conservar el resultado contra el test set de Kaggle y el score obtenido fue de 0.97771 por lo que podemos asumir que fue exitoso.

Además de notar que los trazos no son perfectos, la orientación y rotación de los dígitos no es siempre la misma. Se podría a llegar evaluar un experimento de aplicar rotaciones aleatorias de cierto grado a las imágenes.
