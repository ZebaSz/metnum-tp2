\section{Introducción}

Este trabajo práctico tiene como objetivo el desarrollo y estudio de una herramienta que permita reconocer dígitos manuscritos en imágenes. El algoritmo capaz de llevar esto a cabo es uno de clasificación supervisado que fue entrenado con un lote de caracteres conocido, de modo que le sea posible reconocer otras instancias de esos caracteres aprendidos que no se encuentren en la base de datos de entrenamiento.

En cuanto al reconocimiento de dígitos, estos serán entre el 0 y el 9, y estarán plasmados en imágenes en escalas de grises.

El resultado del algoritmo final fue medido con distintas métricas (HABLAR DE RECALL, PRECISION Y BLABLA).