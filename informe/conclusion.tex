\section{Conclusiones}

En base a lo explicado en las secciones anteriores de este trabajo, pudimos concluir que los valores que mejor funcionan para nuestro algoritmo son 5 para kNN y 30 para el alfa de PCA. Más aún, podemos decir con confianza que el uso de análisis de componentes principales mejoró la calidad y el tiempo de ejecución de forma drástica para el set de datos provisto. No obstante, para que esto funcione de forma óptima se necesita un set de datos amplio como el utilizado para este trabajo. Además, encontramos que el método tiene un límite de precisión que no pudimos superar a pesar de realizar distintas técnicas y experimentaciones, si bien el mismo es elevado (casi 98\% de Accuracy).

En resumen, consideramos que este algoritmo es muy útil en casos donde una exactitud del $100\%$ no es indispensable, dada la calidad de los resultados y su sencilla implementación. Hacemos esta aclaración ya que este mismo algoritmo puede ser y es comunmente utilizado para múltiples problemas de clasificación, donde los datos de entrada son multidimensionales.
